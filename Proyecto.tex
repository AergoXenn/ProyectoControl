% Document Type/Class
\documentclass[12pt,a4paper]{article}

% Packages 
\usepackage[utf8]{inputenc}
\usepackage[T1]{fontenc}
\usepackage[spanish]{babel}
\usepackage{amsmath}
\usepackage{amsfonts}
\usepackage{amssymb}
\usepackage{graphicx}
\usepackage{siunitx}
\usepackage{parskip}
\usepackage{textcomp}
\usepackage{gensymb}
\usepackage{multicol}

\newcommand{\expnum}[2]{{#1}\mathrm{e}{#2}}
\newcommand{\eee}[2]{{#1}\times 10^{#2}}

% Document Options 
\linespread{1.25}
\decimalpoint

\begin{document}
	
	\begin{titlepage}
		\begin{center}
      % Upper margin  
      \vspace{1.5cm}
      
      % Title -> Large + Bold
      \huge
      \textbf{Diseño de un Controlador por el Método del Lugar de las Raíces}
      
      % Vertical space
      \vspace{.5cm}
      
      % So
      \large
      Proyecto Final
      
      % Vertical Space (Tune)
      \vspace{1cm}

      \small
      \textbf{Zaira Lakeisha Rodríguez González} \break
      \textbf{Kevin Fernando Becerra Núñez} \break
      \textbf{Brandon Olaf Contreras Herrera} \break
      \textbf{Luis Fernando Maravilla Valdivia} \break
      \textbf{Diego Alejandro Sánchez Kelly} \break

      % Vertical space (Tune)
      \vspace{0.5cm}
      
      % Institution logo
      \includegraphics[width=0.4\textwidth, keepaspectratio]{./Resources/UDG.png}
      
      % Vertical fill sets the lower margin automatically
      \vfill
      
      \begin{center}
        \textbf{Abstract} \break
        Utilizando el sistema de demostración IP02 de la marca Quanser, se diseñó un compensador de adelanto 
        para el control de la posición de la planta en el demostrador. El diseño se lleva a cabo utilizando 
        la técnica del lugar geométrico de las raíces, usando MATLAB, Simulink y técnicas manuales. 
      \end{center}

      \vspace{0.5cm}

      % Subject & Institution
      \large
      Ingeniería de Control \break
      Universidad de Guadalajara.

      % Lower margin is automatic
		\end{center}
	\end{titlepage}

  % Mandatory pagebreak
	\pagebreak 

  \tableofcontents
  \pagebreak 

  % Optional pagebreak  
	%\pagebreak
	
  % Mandatory pagebreak   
	\pagebreak

	  \section{Introducción} \label{sec:intro}
	
      El diseño del controlador comienza con el modelado matemático de la dinámica del sistema seleccionado. Para 
      facilitar el trabajo, el fabricante se ha encargado de proporcionar un análisis de la dinámica del sistema, 
      a continuación, un resumen del mismo. 

      \subsection{Método del Lugar de las Raíces}

        La intención de este proyecto es la de diseñar un compensador de adelanto por medio del lugar geométrico de las 
        raíces. 

        En algunos sistemas, el simple hecho de variar la ganancia de un sistema en lazo cerrado será suficiente para 
        modificar los polos de lazo cerrado a una ubicación diferente y deseada. 
        
        En la mayoría de sistemas, sin embargo, esto no es suficiente, por lo que es necesario un análisis más profundo 
        sobre las raíces del sistema. Esto permite determinar si es necesario añadir polos o ceros en lazo abierto
        para modificar la respuesta del sistema a una respuesta deseada. 

        \emph{W. R. Evans} diseñó un método conocido como el método de \emph{Root Locus} o
        \emph{Lugar Geométrico de las Raíces}, en el cual las raíces de la ecuación característica de un sistema
        se grafican para todos los valores de un parámetro del sistema. 

        Este parámetro es casi siempre una ganancia \( K_{c} \), del sistema, lo cuál permite observar modificaciones 
        en la ubicación de los polos de lazo cerrado al añadir ceros o polos de lazo abierto. 

        Esto permite encontrar una combinación de ganancia, polos de lazo abierto y ceros de lazo abierto que 
        modifican la respuesta del sistema a la forma deseada. 

      \subsection{Descripción de la Planta}

        La planta de demostración está compuesta por un carro montado en un eje horizontal de acero, así como un 
        eje secundario que permite estabilizar el movimiento del sistema.

        En el eje secundario, el carrito implementa un par de engranajes, que alimentan a un encoder rotatorio de 
        cuadratura o un potenciómetro analógico, según el modelo del dispositivo. 

        El dispositivo, entonces, permite conocer la posición del carrito por medio de 2 diferentes mecanismos, 
        por extensión, esto también permite conocer la velocidad y aceleración del mismo. La Figura~\ref{fig:intro:planta}
        muestra el dispositivo, cortesía del fabricante.

        \begin{figure}
          \centering
          \includegraphics[width=0.65\textwidth, keepaspectratio]{./Resources/IP02_Image.png}
          \caption{Planta}
          \label{fig:intro:planta}
        \end{figure}

        Si bien un motor de corriente directa es, por si mismo, un sistema estable, al añadir el resto de 
        elementos que no necesariamente comparten esta característica, un sistema como este tenderá a ser 
        inestable.

        \pagebreak

	  \section{Modelado}

      Para realizar el control de cualquier tipo de planta o sistema, es primero necesario realizar un modelado 
      matemático del sistema. 
        
      \subsection{Dinámica del Sistema}

        La parte más relevante del sistema es el motor de corriente directa que se encarga de mover el carrito. 
        
        Su modelo matemático es bastante simple en principio, un circuito RLC, sin embargo, como se puede observar en la
        Figura~\ref{fig:modeling:plant_electrical_map}, la salida del motor es una medición angular. 

        Por lo tanto, es necesario primero aplicar un análisis mecánico. Dejemos que \( x\left(t\right) \) sea la 
        posición del carro, tal que, 

        \begin{equation}
          M \left(\dfrac{d^{2}}{dt^{2}}x\left(t\right)\right) + F_{ai}\left(t\right) = 
          F_{c} \left(t\right) - B_{eq}\left(\dfrac{d}{dt}x \left(t\right)\right)
          \label{eq:modeling:initial_dynamic}
        \end{equation}

        donde, \( F_{ai} \) es la inercia experimentada por el carro debido a la rotación del motor y \( F_{c}\) es 
        la fuerza generada por el motor.

        \begin{figure}[h]
          \centering
          \includegraphics[width=0.65\textwidth, keepaspectratio]{./Resources/IP02_Electrical_Model.png}
          \caption{Modelo Eléctrico de la Planta}
          \label{fig:modeling:plant_electrical_map}
        \end{figure}

        Respectivamente, 

        \begin{equation}
          F_{c} = \dfrac{\eta_{g} K_{g} T_{m}}{r_{mp}}
          \label{eq:modeling:motor_generated_force}
        \end{equation}

        y
        
        \begin{equation}
          F_{ai} = \dfrac{\eta_{g} K_{g} T_{ai}}{r_{mp}}
          \label{eq:modeling:armature_inertial_force}
        \end{equation}

        Desde \ref{eq:modeling:motor_generated_force}, entendemos que es posible modelar el motor 
        por medio de las leyes de voltaje de Kirchhoff, así, 

        \begin{equation}
          V_{m} - R_{m} I_{m} - L_{m} \left(\dfrac{\partial}{\partial t} I_{m}\right) - E_{emf} = 0
          \label{eq:modeling:kirchhoff_kvl}
        \end{equation}

        Dado que \( L_{m} \ll R_{m} \) podemos simplificar y despejar para obtener, 

        \begin{equation}
          I_{m} = \dfrac{V_{m} - E_{emf}}{R_{m}}
          \label{eq:modeling:current_kvl}
        \end{equation}
        
        Entonces, 

        \begin{equation}
          F_{c} = \dfrac{ \eta_{g} K_{g} \eta_{m} K_{t} \left(V_{m} - K_{m} \omega_{m}\right)}{R_{m} r_{mp}}
          \label{eq:modeling:motor_force_simp1}
        \end{equation}

        Considerando que \(\omega_{n}\) es la velocidad angular del motor, podemos reescribir el 
        parámetro en función de la velocidad lineal del carro, esto es útil a la hora del diseño del controlador. 

        Así, 

        \begin{equation}
          \omega_{n} = \dfrac{K_{g} \left(\dfrac{d}{dt}x \left(t\right)\right)}{r_{mp}}
          \label{eq:modeling:angular_velocity_relationship}
        \end{equation}

        Que finalmente establece que 

        \begin{equation}
          F_{c} = \dfrac{\eta_{g} K_{g} \eta_{m} K_{t} \left(V_{m} - K_{g} K_{m} \left(\dfrac{d}{dt} x \left(t\right)\right)\right) }{R_{m} {r_{mp}}^{2}}
          \label{eq:modeling:motor_force_final}
        \end{equation}

        Por otro lado, desde \ref{eq:modeling:armature_inertial_force}, sabemos que la fuerza aplicada será una función del torque, 
        por lo tanto, es posible representar \( T_{ai}\) como una función de la posición angular, tal que, 

        \begin{equation}
          T_{ai} \left(t\right) = J_{m} \left(\dfrac{d^{2}}{dt^{2}} \theta_{m} \left(t\right)\right)
          \label{eq:modeling:armature_torque_simp1}
        \end{equation}

        Las propiedades del sistema establecen que, 

        \begin{equation}
          \theta_{m} = \dfrac{K_{g} x}{r_{mp}}
          \label{eq:modeling:gearbox_angular_relationship}
        \end{equation}

        Por lo tanto, sustituyendo, 

        \begin{equation}
          T_{ai} \left(t\right) = J_{m} \left(\dfrac{d^{2}}{dt^{2}} \dfrac{K_{g} x}{r_{mp}} \left(t\right)\right)
          \label{eq:modeling:armature_torque_final}
        \end{equation}

        \begin{equation}
          F_{ai} = \dfrac{\eta_{g} {K_{g}}^{2}  J_{m} \left(\dfrac{d^{2}}{dt^{2}} \dfrac{K_{g} x}{r_{mp}} \left(t\right)\right)}{{r_{mp}}^{2}}
          \label{eq:modeling:armature_force_final}
        \end{equation}

        Finalmente, sustituyendo las ecuaciones \ref{eq:modeling:motor_force_final} y \ref{eq:modeling:armature_force_final}
        en la Ecuación~\ref{eq:modeling:initial_dynamic} y reorganizando, 

        \begin{multline}
          \left( M + \dfrac{\eta_{g} {K_{g}}^{2}}{{r_{mp}}^{2}} J_{m} \right) \left( \dfrac{d^{2}}{dt^{2}} x \left(t\right) \right) + \\
          \left( B_{eq} + \dfrac{\eta_{g} {K_{g}}^{2} \eta_{m} K_{t} K_{m}}{R_{m} {r_{mp}}^{2}} \right) \left(\dfrac{d}{dt} x\left(t\right)\right) = \\
          \dfrac{\eta_{g} K_{g} \eta_{m} K_{t} K_{m}}{R_{m} r_{mp}}
          \label{eq:modeling:dynamic_equation}
        \end{multline}

      \subsection{Función de Transferencia}

        Obtenemos la función de transferencia genérica de lazo abierto de la planta aplicando la transformada de Laplace a 
        la Ecuación~\ref{eq:modeling:dynamic_equation}. Así, 
      
        \begin{equation}
          G\left(s\right) = \frac{r_{mp} \eta_{g} K_{g} \eta_{m} K_{t}}{s^{2} \left(R_{m} M r_{mp}^{2} \eta_{g} K_{g}^{2} J_{m}\right) + 
          s\left(\eta_{g} K_{g}^{2} \eta_{m} K_{t} K_{m} + B_{eq} R_{m} r_{mp}^{2}\right)}
          \label{eq:modeling:plant_generic_tf}
        \end{equation}

        Para obtener la función de transferencia final, se sustituyen en la Ecuación~\ref{eq:modeling:plant_generic_tf}
        los parámetros físicos proporcionados por el fabricante, ubicados en la 
        Figura~\ref{fig:modeling:plant_parameters_01} y la Figura~\ref{fig:modeling:plant_parameters_02}.

        Al sustituír se obtiene, 

        \begin{equation}
          G\left(s\right) = \frac{\expnum{1.8063}{-4}}{s^{2}\left(\expnum{6.8472}{-5}\right) + s\left(\expnum{1.2605}{-3}\right)}
          \label{eq:modeling:plant_tf}
        \end{equation}

        Esto se puede simplificar posteriormente, tal que,  
        
        \begin{equation}
          G\left(s\right) = \frac{2.638}{s^{2} + s\left(18.409\right)}
          \label{eq:modeling:plant_tf_final}
        \end{equation}

        La anterior operación no modifica la ubicación de los polos y ceros del controlador, sin embargo, 
        si permite que los cálculos posteriores sean más sencillos. 

        \begin{figure}
          \centering
          \includegraphics[width=0.65\textwidth, keepaspectratio]{./Resources/IP02_Parameters_01.png}
          \caption{Parámetros Físicos de la Planta, 1 de 2. }
          \label{fig:modeling:plant_parameters_01}
        \end{figure}

        \begin{figure}
          \centering
          \includegraphics[width=0.65\textwidth, keepaspectratio]{./Resources/IP02_Parameters_02.png}
          \caption{Parámetros Físicos de la Planta, 2 de 2. }
          \label{fig:modeling:plant_parameters_02}
        \end{figure}

    \subsection{Polos y Ceros de Lazo Abierto}

      \begin{figure}
        \centering
        \includegraphics[width=0.75\textwidth, keepaspectratio]{./Resources/IP02_OpenLoop_rlocus.jpg}
        \caption{Lugar geométrico de las raíces, Lazo Abierto}
        \label{fig:modeling:open_loop_root_locus}
      \end{figure}

      De la Ecuación~\ref{eq:modeling:plant_tf_final}, se obtiene la ecuación característica del sistema, 
      así, 

        \begin{equation*}
          \begin{aligned}
            s^{2}+ s\left(18.4092\right) &= 0 \\
            s^{2} &= -s\left(18.4092\right)\\
          \end{aligned}
          \label{eq*:modeling:roots}
        \end{equation*}    

        de lo anterior, \(s \rightarrow \left\{0, -18.4092\right\}\) en el caso de los polos. 

        Observando el numerador de la función original, es evidente que no existen ceros, por lo que no es necesario
        realizar el cálculo. 

        Sin embargo, utilizando el comando \verb|rlocus| en MATLAB, se puede obtener una gráfica del lugar geométrico
        de las raíces por medio de la cual podemos verificar los cálculos anteriores. 

        Así, en la Figura~\ref{fig:modeling:open_loop_root_locus}, se comprueba que no existen ceros en la función, 
        y que ambos polos se encuentran en su ubicación esperada.

    \pagebreak

    \section{Análisis de lazo abierto}

      Previo a comenzar el diseño del controlador, es necesario conocer la respuesta del sistema a diferentes 
      estímulos o perturbaciones. Esto es relevante ya que nos permitirá determinar el tipo de controlador 
      y sus parámetros de diseño. 

      Utilizamos el método del lugar geométrico de las raíces, por lo que es necesario realizar el análisis en el 
      dominio del tiempo. 

      \subsection{Respuesta a Escalón Unitario}

        Utilizando el comando \verb|step| en MATLAB, es posible obtener la respuesta en el tiempo del sistema 
        a un estímulo escalón.   

        Como se puede observar en la Figura~\ref{fig:response:open_loop_step_response}, la respuesta en lazo abierto 
        del sistema es característica de un sistema inestable. 

        \begin{figure}
          \centering
          \includegraphics[width=0.75\textwidth, keepaspectratio]{./Resources/IP02_OpenLoop_Step.jpg}
          \caption{Respuesta Escalón Unitario, Lazo Cerrado}
          \label{fig:response:open_loop_step_response}
        \end{figure}

        Se denomina inestable al sistema debido a que un estímulo constante causa que el sistema responda
        de manera extrema e incontrolable. 

        Por lo tanto, es necesario diseñar un controlador o compensador que permita corregir la respuesta del 
        sistema. En este caso, la respuesta escalón es de interés ya que se busca controlar velocidad y posición 
        del carro. 

  \pagebreak


  \section{Diseño del Compensador}

      Un compensador de adelanto tiene la intención de cambiar la respuesta del sistema introduciendo un polo y un cero. Esto 
      tiene el efecto de mover el LGR para hacerlos pasar por una serie de polos deseados. 

      Posteriormente, se sintoniza el controlador seleccionando la ganancia adecuada para la cuál la condición de ángulo 
      se cumple. 

      Los compensadores, ya sean de adelanto, retardo o adelanto-retardo, no son controladores propiamente, sin embargo, permiten
      implementar acciones de control de forma mucho más sencilla que, por ejemplo, un controlador PID. 

      A continuación se proporciona contexto sobre el proceso, fórmulas y técnicas de diseño utilizadas para 
      el diseño final del controlador.

      \subsection{Marco Teórico}

        La teoría de control proporciona una serie de herramientas para moldear la respuesta de un sistema según 
        lo requerido por el diseñador. 

        Para un sistema de segundo orden, (ver Ecuación~\ref{eq:modeling:plant_tf_final}), la forma normal del sistema 
        tiene la siguiente ecuación estándar, 

        \begin{equation}
          G\left(s\right) = \frac{K_{dc}\omega_{n}^{2}}{s^{2} + 2\zeta\omega_{n}s + \omega_{n}^{2}}
          \label{eq:design:2ndorder_std}
        \end{equation}

        con polinomio característico, 

        \begin{equation}
          s^{2} + 2\zeta\omega_{n}s + \omega_{n}^{2}
          \label{eq:design:2ndorder_std_charact}
        \end{equation}

        También sabemos que los parámetros de sobredisparo y tiempo pico se pueden calcular con las siguientes ecuaciones, 
        respectivamente, 

        \begin{equation}
          P_{osht} = 100\exp\left(-\frac{\zeta\pi}{\sqrt[2]{1-\zeta^{2}}}\right)
          \label{eq:design:overshoot_prcntg}
        \end{equation}

        \begin{equation}
          t_{p} = \frac{\pi}{\omega_{n}\sqrt{1-\zeta^{2}}}
          \label{eq:design:peak_time}
        \end{equation}

      \subsection{Parámetros de diseño}

        Se proponen los siguientes parámetros de diseño,

        \begin{description}
          \item[Overshoot] El sistema debe contar con un porcentaje de sobredisparo de máximo un 7\%. 
          \item[Peak Time] El sistema debe alcanzar su punto pico en $ 110\si{\milli\second} $. 
        \end{description}
      
      \subsection{Ubicación de los Polos Deseados}

        Desde \ref{eq:design:overshoot_prcntg}, 

        \begin{equation}
          \zeta = \frac{-\ln\left(P_{osht}\right)}{\sqrt{\pi^{2} + \ln^{2}\left(P_{ovsht}\right)}}
        \end{equation}

        y, desde \ref{eq:design:peak_time}, 

        \begin{equation}
          \omega_{n} = \frac{\pi}{t_{p}\sqrt{1-\zeta^{2}}}
        \end{equation}
        
        entonces, calculando los valores de \( \zeta \)  y de \( \omega_{n} \), obtenemos, 

        \begin{equation*}
          \begin{aligned}
            \zeta &= \frac{-\ln\left(0.08\right)}{\sqrt{\pi^{2} + \ln^{2}\left(0.08\right)}} \\
            \zeta &= 0.64608 \\
          \end{aligned}
        \end{equation*}

        y, 

        \begin{equation*}
          \begin{aligned}
            \omega_{n} &= \frac{\pi}{0.11\sqrt{1-\zeta^{2}}} \\ 
            \omega_{n} &= 37.42218 \\
          \end{aligned}
        \end{equation*}
        
        Despejando \( s \) en la Ecuación~\ref{eq:design:2ndorder_std_charact}, se obtiene que, 

        \begin{equation}
          s = -\zeta\omega_{n} \pm \omega_{n}\sqrt{1 - \zeta^{2}}j
          \label{eq:design:charact_sval}
        \end{equation}

        y, posteriormente, sustituyendo los valores de \( \zeta \)  y \( \omega_{n} \) calculados anteriormente, 
        obtenemos,

        \begin{equation}
          \centering
          \begin{aligned}
            \hat{s} &= \left(-0.64608\right) \left(37.42218\right) \pm \left(37.42218\sqrt{1-0.64608^{2}}\right) \\
            \hat{s} &= -24.17772 \pm 32.69379j
          \end{aligned}
        \end{equation}

        Que es la combinación compleja conjugada de los polos deseados para el compensador de adelanto. 

      \subsection{Condición de Ángulo}

        Conociendo la ubicación deseada del polo, el primer paso es entonces verificar si este ya se encuentra en 
        el lugar geométrico de las raíces, lo cual requiere de aplicar la condición de ángulo. 

        La condición de ángulo establece que, 

        \begin{equation}
          \Sigma_{\angle_{poles}} + \Sigma_{\angle_{zeros}} = 180\degree
          \label{eq:design:angle_condition} 
        \end{equation}

        Calculando,
        
        \begin{equation} 
          \alpha_{1} = 180\degree - tan^{-1}\left(\dfrac{32.69379}{24.17772}\right) 
        \end{equation}

        y 

        \begin{equation} 
          \alpha_{2} = 180\degree - tan^{-1}\left(\dfrac{32.69379}{24.17772 - 18.40923}\right) 
        \end{equation}

        lo cual resulta en \( \alpha_{1} = 126.48367\degree \) y \( \alpha_{2} = 100.00627 \).

        Podemos calcular el ángulo de deficiencias, 

        \begin{equation}
          \phi = 180\degree - \alpha_{1} - \alpha_{2} 
        \end{equation}

        entonces, 

        \begin{equation} 
          \phi =  180\degree - 126.48367\degree - 100.00627\degree 
        \end{equation}

        por lo tanto, \( \phi = 46.48994\degree \).

        Este ángulo, \( \phi \), corresponde al ángulo que deberá proporcionar el compensador para lograr que 
        el lugar geométrico de las raíces pase por el polo deseado. 

      \subsection{Polos y ceros del compensador}
        \label{subsec:polezero}

        Calulamos la sintonización, 

        \begin{equation} 
          \dfrac{\phi}{2} = 23.24497\degree 
        \end{equation}

        por lo tanto, 

        \begin{equation} 
          \alpha_{3} = \dfrac{\alpha_{1}}{2} = \dfrac{126.48367}{2}  
        \end{equation}

        que es, \( \alpha_{3} = 63.2418 \). También, 

        \begin{equation} 
          \alpha_{4} = \alpha_{3} - \dfrac{\phi}{2} = 63.2418 - \dfrac{46.4894}{2} 
        \end{equation}
        
        que es \( 39.9968 \). 

        Entonces, 

        \begin{equation} 
          \tan\left(39.9968\right)\dfrac{32.6937}{d} 
        \end{equation}

        Calculando \( d \), 

        \begin{equation}
           d = \dfrac{32.6937}{\tan\left(39.9968\right)} = 38.9672 
        \end{equation}

        A partir de lo anterior, es posible calcular el polo compensado, 
        
        \begin{equation} 
          \dfrac{1}{\alpha } = 38.9672 + 24.1777 = 69.1449 
        \end{equation}

        similarmente, podemos calcular el cero compensado, 

        \begin{equation} 
          \tan\left(3.5131 \right) = \dfrac{y}{32.69379} 
        \end{equation}

        despejando, 

        \begin{equation} 
          y = 32.6937 \cdot \tan\left(3.5131\right) = 2.0071 
        \end{equation}

        Finalmente, 

        \begin{equation} 
          \dfrac{1}{T} = 24.1777 + 2.0071 = 26.1849
        \end{equation}

        A partir de esto, es posible proceder con la implementación del controlador y la obtención de la función 
        de transferencia final. 
	
    \subsection{Compensador}
    
      La forma estándar de un compensador de adelanto es la siguiente, 

      \begin{equation}  
        G_{c}\left(s\right) = K_{c}\dfrac{s+\dfrac{1}{T}}{s + \dfrac{1}{\alpha T}}
        \label{eq:devel:leadcomp}
      \end{equation}

      donde \( \alpha \) debe encontrarse en la región \( 0 < \alpha < 1 \). 

      Según lo establecido en la Sección~\ref{subsec:polezero}, sustituímos en la 
      Ecuación~\ref{eq:devel:leadcomp} para obtener, 

      \begin{equation} 
        G_{c}\left(s\right) = K_{c}\dfrac{s + 26.18491}{s + 63.1449}
      \end{equation}
          
        \subsubsection{Ganancia del Compensador}

          Desde, 

          \begin{equation}
            0 = 1 + G_{c}\left(s\right)G\left(s\right)
            \label{eq:devel:compdsys_formula}
          \end{equation}

          sustituímos, 

          \begin{equation} 
            1 + K_{c}\left(\dfrac{s + 26.1849}{s + 63.1449}\right)\left(\dfrac{2.6389}{s\left(s + 18.4092\right)}\right) = 0 
          \end{equation}

          y, posteriormente, evaluamos en \( \hat{s} \), 

          \begin{equation} 
            K_{c}\left|\left(\dfrac{s + 26.1849}{s + 63.1449}\right)\left(\dfrac{2.6389}{s\left(s + 18.4092\right)}\right)\right| = 1 
          \end{equation}

          tal que, 

          \begin{equation}
            K_{c} = 794.39503 
          \end{equation}

          Al proceso anterior se le conoce como la \emph{condición de magnitud}. 

      \subsection{Compensador Finalizado}

        Finalmente, 

        \begin{equation} 
          G_{c}\left(s\right) = 794.39503\dfrac{s + 26.18491}{s + 63.1449}
        \end{equation}

      \subsection{Sistema Compensado en Lazo Abierto}

        Finalmente, 

        \begin{equation}
          G_{c}\left(s\right)G_{s}\left(s\right) =  794.39503 \left(\dfrac{s + 26.1849}{s + 63.1449}\right)\left(\dfrac{2.6389}{s\left(s + 18.4092\right)}\right) 
          \label{eq:design:OpenLoop_compensated_tf}
        \end{equation}

        Simplificando, 

        \begin{equation}
          G_{c}\left(s\right)G_{s}\left(s\right) = 2096.3 \dfrac{s + 26.185}{s \left( s + 18.409 \right) \left(s + 63.145\right)}
          \label{eq:design:OpenLoop_compensated_tf_final}
        \end{equation}

  \pagebreak

  \section{Sistema Compensado en Lazo Cerrado}
      
    Comenzamos por aplicar retroalimentación negativa a la función de transferencia del sistema compensado. 

    Este proceso se realiza utilizando el comando \verb|feedback(planta, 1, -1)| en MATLAB, que aplica 
    una retroalimentación con ganancia de \( -1 \) a la planta compensada. 

    El cálculo sigue la fórmula,

    \begin{equation}
      \dfrac{Y\left(s\right)}{X\left(s\right)} = \dfrac{G_{p}\left(s\right)}{1 + G_{p}\left(s\right) G_{c}\left(s\right)}
      \label{eq:}
    \end{equation}
    
    y resulta en, 

    \begin{equation}
      G_{p}\left(s\right)G_{c}\left(s\right) = 2096.3 \dfrac{s + 26.185}{s \left( s + 33.21 \right) \left(s^{2} + 48.35 + 1653\right)}
    \end{equation}

    con polos ubicados en \( s \longrightarrow \left\{-24.1736 \pm 32.6948 j, -33.2071\right\} \),  
    y ceros ubicados en \( s \longrightarrow \left\{-26.1894\right\} \).

    La ubicación de estos polos y ceros se puede comprobar en la Figura~\ref{fig:compd_system:closed_loop_root_locus}. 

    \begin{figure}
      \centering
      \includegraphics[width=0.75\textwidth, keepaspectratio]{./Resources/IP02_ClosedLoop_rlocus.jpg}
      \caption{Lugar geométrico de las raíces, Lazo Cerrado}
      \label{fig:compd_system:closed_loop_root_locus}
    \end{figure}

    Ya que el controlador se encuentra sintonizado, el uso del lugar geométrico de las raíces probablemente no es 
    el más adecuado, por lo que la ganancia no se modificaría. 

    Sin embargo, se decidió integrar esta gráfica ya que el movimiento de las raíces permite visualizar más 
    claramente la nueva ubicación de los polos y ceros de lazo cerrado en el sistema. 

    \subsection{Análisis de lazo cerrado}

      Finalmente, es momento de verificar la funcionalidad del compensador en lazo cerrado, para ello, al igual 
      que como se hizo en lazo abierto, se aplica un estímulo escalón unitario al sistema. 

      \begin{figure}
        \centering
        \includegraphics[width=0.75\textwidth, keepaspectratio]{./Resources/IP02_ClosedLoop_step.jpg}
        \caption{Respuesta escalón, Lazo Cerrado}
        \label{fig:compd_system:closed_loop_step}
      \end{figure}

    Como se puede observar en la Figura~\ref{fig:compd_system:closed_loop_step}, la respuesta a un estímulo escalón 
    mejora claramente desde una respuesta totalmente inestable a una respuesta característica de segundo orden con 
    poco sobredisparo, un corto tiempo de subida y un corto tiempo de asentamiento, de \( 16 \)\% y 
    \( 30\si{\milli\second} \) y \( 146\si{\milli\second} \) respectivamente.

    Comparando esto con los parámetros definidos originalmente, de \( 110\si{\milli\second} \) para el tiempo de asentamiento
    y \( 7 \)\% para el porcentaje de sobredisparo, es claro que los parámetros deseados no se cumplen. 

    Considerando posible error humano, así como la naturaleza del diseño de compensadores, no es una sorpresa que la respuesta 
    no sea precisamente la deseada. 

    Sin embargo, la mejora es sustancial y suficiente para el requerimiento de la aplicación. 

\end{document}