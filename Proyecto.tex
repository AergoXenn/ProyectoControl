% Document Type/Class
\documentclass[12pt,a4paper]{article}

% Packages 
\usepackage[utf8]{inputenc}
\usepackage[T1]{fontenc}
\usepackage[spanish]{babel}
\usepackage{amsmath}
\usepackage{amsfonts}
\usepackage{amssymb}
\usepackage{graphicx}
\usepackage{siunitx}
\usepackage{parskip}

% Document Options 
\linespread{1.25}
\decimalpoint

\begin{document}
	
	\begin{titlepage}
		\begin{center}
      % Upper margin  
      \vspace{2cm}
      
      % Title -> Large + Bold
      \huge
      \textbf{Diseño de un Controlador por el Método del Lugar de las Raíces}
      
      % Vertical space
      \vspace{.5cm}
      
      % So
      \large
      Proyecto Final
      
      % Vertical Space (Tune)
      \vspace{2cm}

      \small
      \textbf{Zaira Lakeisha Rodríguez González} \break
      \textbf{Kevin Fernando Becerra Núñez} \break
      \textbf{Brandon Olaf Contreras Herrera} \break
      \textbf{Luis Fernando Maravilla Valdivia} \break
      \textbf{Diego Alejandro Sánchez Kelly} \break

      % Vertical space (Tune)
      \vspace{1cm}
      
      % Institution logo
      \includegraphics[width=0.4\textwidth, keepaspectratio]{./Resources/UDG.png}
      
      % Vertical fill sets the lower margin automatically
      \vfill
      
      % Subject & Institution
      \large
      Ingeniería de Control \break
      Universidad de Guadalajara.

      % Lower margin is automatic
		\end{center}
	\end{titlepage}

  % Mandatory pagebreak
	\pagebreak 
	
    \begin{center}
      \textbf{Abstract} \break
      Utilizando el sistema de demostración IP01/IP02 de la marca Quanser, se diseñó un controlador de posición 
      para una planta que consiste en un carro con masa montado en un eje lineal. El diseño del controlador se llevó 
      por medio del método del lugar de las raíces. 
    \end{center}

  % Optional pagebreak  
	%\pagebreak
	
    \tableofcontents
	
  % Mandatory pagebreak   
	\pagebreak
	
	  \section{Introducción} \label{sec:intro}
	
      El diseño del controlador comienza con el modelado matemático de la dinámica del sistema seleccionado. Para 
      facilitar el trabajo, el fabricante se ha encargado de proporcionar un análisis de la dinámica del sistema, 
      a continuación, un resumen del mismo. 

      \subsection{Análisis}

        La planta de demostración está compuesta por un carro montado en un eje horizontal de acero, así como un 
        eje secundario que permite estabilizar el movimiento del sistema.

        En el eje secundario, el carrito implementa un par de engranajes, que alimentan a un encoder rotatorio de 
        cuadratura o un potenciómetro analógico, según el modelo del dispositivo. 

        El dispositivo, entonces, implementa

	    \subsection{Conceptos}
	
	    \subsection{Experimento} \label{subsec: EXP}
	
	\pagebreak
	
	\section{Desarrollo}
	
	\subsection{Teoría Experimental.}
	
	%Que se espera del experimento y la teoría detrás de el.

	\subsubsection{ Mediciones.}

	%Resultados preeliminares y demás formas.
	
	\subsection{Fuentes de error.}
	
	%Justificacion de resultados si se encuentran fuera de lo esperado y explicación si son congruentes.
	
	\pagebreak

	\section{Resultados.} 
	
	%Exposición de lo obtenido de las mediciones y experimentos.
	
	\pagebreak
	
	\section{Discusión.}
	
	\subsection{Análisis de los resultados.}
	
	%Explicación de los resultados y consideraciones sobre los errores y comparación de los resultados contra lo esperado y lós fenómenos reales.
	
	\subsection{Aprendizajes.}
	
	%Exposición del conocimiento obtenido, sea por parte investigativa o por conjeturas de los resultados que fueron comprobadas o descartadas durante el experimento.
	
	\pagebreak
	
	\section{Conclusiones.}
	 
	%Como la discusión encaja con las concepciones previas. Menciona utilidad de la práctica o experimento y relevancia en cuanto a los conocimientos del curso o carrera.
	
	\pagebreak
	
	\appendix
	
	\section{Apéndice 1}
	
	%Códigos, plots, imágenes, etc.
	
	\section{Apéndice 2}
	
	%Códigos, plots, imágenes, etc.
	
\end{document}